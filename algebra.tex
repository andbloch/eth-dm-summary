
\textbf{How to compute modular inverses with gcd}

\todo{How is this done?}

We're looking for the solution $x=a^{-1}$ of
\[
ax\equiv_m=1
\]
Lets transform this, and rename $x$ to $u$ and let $v\in \Z$
\[
au + mv = 1 \Rightarrow au+mv\equiv_m au = 1
\]
Hence if we solve, $\gcd(m,a)$ (since $m>a$) then our solution is $x=u=a^{-1}$.


%%%%%%%%%%%%%%%%%%%%%%%%%%%%%%%%%%%%%%%%%%%%%%%%%%%%%%%%%%%%%%%%%%%%%%%%%%%%%%%%
%%%%%%%%%%%%%%%%%%%%%%%%%%%%%%%%%%%%%%%%%%%%%%%%%%%%%%%%%%%%%%%%%%%%%%%%%%%%%%%%
%%%%%%%%%%%%%%%%%%%%%%%%%%%%%%%%%%%%%%%%%%%%%%%%%%%%%%%%%%%%%%%%%%%%%%%%%%%%%%%%
%%%%%%%%%%%%%%%%%%%%%%%%%%%%%%%%%%%%%%%%%%%%%%%%%%%%%%%%%%%%%%%%%%%%%%%%%%%%%%%%
%%%%%%%%%%%%%%%%%%%%%%%%%%%%%%%%%%%%%%%%%%%%%%%%%%%%%%%%%%%%%%%%%%%%%%%%%%%%%%%%
%%%%%%%%%%%%%%%%%%%%%%%%%%%%%%%%%%%%%%%%%%%%%%%%%%%%%%%%%%%%%%%%%%%%%%%%%%%%%%%%
%%%%%%%%%%%%%%%%%%%%%%%%%%%%%%%%%%%%%%%%%%%%%%%%%%%%%%%%%%%%%%%%%%%%%%%%%%%%%%%%
%%%%%%%%%%%%%%%%%%%%%%%%%%%%%%%%%%%%%%%%%%%%%%%%%%%%%%%%%%%%%%%%%%%%%%%%%%%%%%%%
%%%%%%%%%%%%%%%%%%%%%%%%%%%%%%%%%%%%%%%%%%%%%%%%%%%%%%%%%%%%%%%%%%%%%%%%%%%%%%%%

\section{Algebra Notes}

\Def{*} Let $S_n$ be the set of $n!$ permutations of $n$ elements, i.e., the
set of bijections $\{1,\ldots n\}\to \{1,\ldots,n\}$. The group
$\alg{S_n;\circ,^{-1},\id}$ is called the \emph{symmetric group} of $n$
elements. $S_n$ is non-abelian for $n\geq 3$.

\Com Associativity is a very special property of an operation, but it is of
crucial importance in algebra. Associativity of $*$ means that the element
$a_1*a_2*a_3*\cdots*a_n$ is uniquely defined, independent of the order in which
elements are combined through $*$. This also justifies the use of the notation
$\sum_{i=1}^{n}$ if the operation $*$ is called addition, or $\prod_{i=1}^{n}$
if the operation $*$ is called multiplication.

\textbf{Examples}

\Ex The set of even integers $E$ forms a semigroup (or magma) with respect to
multiplication: $\alg{E;\cdot}$. - There is no neutral element.

\Ex The set uf functions $A\to A$ form a monoid with respect to function
composition: $\alg{A^{A};\circ,\id}$.

\Com To prove the uniqueness of the invers (if it exists), we need $*$ to be
associative:

\Ex Consider again $\alg{A^{A};\circ,\id}$. A function $f\in A^{A}$ has a left
inverse only if it is injective, and it has a right inverse only if it is
surjective. Hence it has only an inverse $f^{-1}$ if and only if $f$ is
bijective. In this case, $f\circ f^{-1} = f^{-1}\circ f = \id$.

\Com We can write $\alg{G;*,\widehat{\ },e}$ instead of $\alg{G;*}$ if we want
to make the inverse operateion an dthe neutral element explicit. Also, often one
simply writes $G$ instead of $\alg{G;*}$ if $*$ is understood. If the operation
$*$ is called addition $(+)$ [multiplication $(\cdot)$], then the inverse of $a$
is denoted $-a$ [$a^{-1}$ or $1/a$] and the neutral element is denoted $0$
[$1$].

\Ex Homomorphism: Projection in $\R^3$ onto a plane and addition.

\Ex Homomorphism: Det. for $n\times n$-Matrices and multiplication.

\Ex The set of symmetries and rotations, denoted $S_\square$, constitutes a
subgroup (with 8 elements) of the set of 24 permutations on $4$ elements.

\Ex The characteristic of $\alg{\Z_m;\oplus,\ominus,0,\cdot,1}$ is $m$. The
characteristic of $\Z$ is 0.

\Ex Set of units of rings: $\Z^*=\{-1,1\}$, $\R^*=\R-\{0\}$ $\C^*=\{1,-1,i,-i\}$.

\Ex Any integer not relatively prime to $m$ is a zerodivisor of $\Z_m$. The set
of units of $\Z_m$ is $\Z^*_m$ (Definition 7.18). A special property of $\Z_m$
is that each non-zero element is either a unit or a zero-divisor.

\Ex $\Z,\Q,\R$, and $\C$ are integral domains.

\Com A polynomial is a formal expression. It can, but need not, be considered as
a function $R\to R$.

\Com The interpretation of ``minus infinity'' is that it is a quantity which
remains unchanged when an arbitrary integer is added to it.

\Ex Fields: $\Q,\R,\C$

\Ex Not Fields: $\Z$ and $R[x]$ for any ring $R$.

\Com Fields are of crucial importance because in a field one can not only add,
subtract, and multiply, but one can also divide by any nonzero element. This is
the abstraction underlying many algorithms like those for solving systems of
linear equations (e.g. by Gaussian elimination) or for polynomial interpolation.
Also, a vector space, a crucial concept in mathematics, is defined over a field,
the so-called base field. Vector spaces over $\R$ are just a special case.


%%%%%%%%%%%%%%%%%%%%%%%%%%%%%%%%%%%%%%%%%%%%%%%%%%%%%%%%%%%%%%%%%%%%%%%%%%%%%%%%
%%%%%%%%%%%%%%%%%%%%%%%%%%%%%%%%%%%%%%%%%%%%%%%%%%%%%%%%%%%%%%%%%%%%%%%%%%%%%%%%
%%%%%%%%%%%%%%%%%%%%%%%%%%%%%%%%%%%%%%%%%%%%%%%%%%%%%%%%%%%%%%%%%%%%%%%%%%%%%%%%
%%%%%%%%%%%%%%%%%%%%%%%%%%%%%%%%%%%%%%%%%%%%%%%%%%%%%%%%%%%%%%%%%%%%%%%%%%%%%%%%
%%%%%%%%%%%%%%%%%%%%%%%%%%%%%%%%%%%%%%%%%%%%%%%%%%%%%%%%%%%%%%%%%%%%%%%%%%%%%%%%
%%%%%%%%%%%%%%%%%%%%%%%%%%%%%%%%%%%%%%%%%%%%%%%%%%%%%%%%%%%%%%%%%%%%%%%%%%%%%%%%
%%%%%%%%%%%%%%%%%%%%%%%%%%%%%%%%%%%%%%%%%%%%%%%%%%%%%%%%%%%%%%%%%%%%%%%%%%%%%%%%
%%%%%%%%%%%%%%%%%%%%%%%%%%%%%%%%%%%%%%%%%%%%%%%%%%%%%%%%%%%%%%%%%%%%%%%%%%%%%%%%
%%%%%%%%%%%%%%%%%%%%%%%%%%%%%%%%%%%%%%%%%%%%%%%%%%%%%%%%%%%%%%%%%%%%%%%%%%%%%%%%


\section{Algebra}

\textbf{General Problem-Solving Approach}
\begin{enumerate}
  \item Eine Seite der zu beweisenden Aussage aufschreiben
  \item Neutrales Element mit Operation an nützlicher Stelle hinzufügen
  \item Neutrales Element geschickt durch anderes bekanntes, das nützlich sein
  könnte ausdrücken
  \item Assoziativität von $*$ oder andere Gruppenaxiome ausnützen
  \item Axiome anwenden, biss gewünschtes auf der rechten Seite steht, q.e.d.
\end{enumerate}

\subsection{Introduction}

\Def[Operation] An \emph{operation} on a set $S$ is a function $S^n\to S$, where
$n\geq 0$ is called the ``\emph{arity}'' of the operation.

% TODO: useful comment
\begin{comment}
\Com Operations with arity $1$ and $2$ are called unary and binary operations,
respectively. An operation with arity 0 is called a constant (or nullary); it is
a fixed element from the set $S$. In many cases, only binary operations are
actually listed explicitly.
\end{comment}

\Def[Algebra] An \emph{algebra} or (\emph{algebraic system} or $\Omega$-algebra)
is a pair $\alg{S;\Omega}$ where $S$ is a set (the \emph{carrier} of the
algebra) and $\Omega=(\omega_1,\ldots,\omega_n)$ is a list of operations on $S$.

% TODO: useful comment
\begin{comment}
There are three levels of abstraction at which one can study algebras.
\begin{itemize}
  \item The \emph{concrete} level: One studies concrete structures and their
  properties.
  \item The \emph{axiomatic} level: One studies a certain class of algebras
  specified by a set of axioms. The axioms are seen as postulates assumed to be
  satisfied for all algebras under consideration. Equivalently, one considers
  all algebras satisfying these axioms, not necessarily with a concrete example
  in mind. Consequences derived from the axioms hold for all algebras in the
  general class of algebras satisfying the axioms.
  \item The \emph{universal} level: One studies algebras without specifying the
  aximos nor the type. For instances, subalgebra, isomorphism, and the direct
  product (see later) are universal algebraic concepts that apply to any
  algebra.
\end{itemize}

Usually, algebra is treated at the abstract axiomatic level, giving examples at
the concrete level. The universal level is considered less frequently.
\end{comment}

\subsection{Semigroups, Monoids, Groups}

(See overview)

\subsection{Homomorphisms and Isomorphisms}

\Def[Homomorphism] A function $\psi$ from a group $\alg{G;*,\widehat{ \ }, e_G}$
to a group $\alg{H;\star,\widetilde{\ }, e_H}$ is a \emph{group homomorphism}
if, for all $a,b\in G$,
\[
\psi(a*b)=\psi(a)\star\psi(b).
\]
\Def[Isomorphism] If $\psi$ is a bijection from $G$ to $H$, then it is called an
\emph{isomorphism}.

\Com Homomorphism $\approx$ structure-preserving function from an algebraic
structure into another algebraic structure

\Lem{7.5} A group homomorphism $\psi$ from $\alg{G;*,\widehat{ \ }, e_G}$ to 
$\alg{H;\star,\widetilde{\ }, e_H}$ satisfies
\begin{enumerate}[label=(\roman*)]
  \item $\psi(e_G) = e_H$,
  \item $\psi(\widehat{a})=\widetilde{\psi(a)}$ for all $a$.
\end{enumerate}




\vspace{100pt}


\subsection*{7.3 The Structure of Groups}

\Def{7.13} A subset $H$ of a group $\alg{G;*,\widehat{ \ }, e}$ is called a
\emph{subgroup} of $G$ if $\alg{H;*,\widehat{ \ }, e}$ is a group, i.e. if $H$
is closed with respect to all operations:
\begin{enumerate}[label=(\arabic*)]
  \item $a*b\in H$ for all $a,b \in H$,
  \item $e\in H$, and
  \item $\widehat{a}\in H$ for all $a \in H$.
\end{enumerate}

\Com For any group $G$ there exist two trivial subgroups: the subset $\{e\}$ and
$G$ itself.

\sep

\todo{In the remainder of this section we will use a multiplicative notation}

\sep

\Def{*} For $n\in \Z$, $a^{n}$ is defined recursively:
\begin{itemize}
  \item $a^0=e$,
  \item $a^n = a\cdot a^{n-1}$ for $n\geq 1$, and
  \item $a^{n} = (a^{-1})^{|n|}$ for $n\leq -1$.	
\end{itemize}

It is easy to see that for all $m,n\in \Z$
\[
a^m\cdot a^n = a^{m+n} \quad \text{and} \quad (a^m)^n = a^{mn}.
\]

\sep

\Def{7.14} Let $G$ be a group and let $a$ be an element of $G$. The \emph{order}
of $a$, denoted $\ord(a)$, is the least $m>1$ such that $a^{m}=e$, if such an
$m$ exists, and $\ord(a)=\infty$ otherwise.

By definition, $\ord(e)=1$.

If $\ord(a)=2$ for some $a$, then $a^{-1}=a$; such
an $a$ is called self-inverse.

\Ex The order of $6$ in $\alg{\Z_{20};\oplus,\ominus,0}$ is $10$. This can be
seen easily since $60=10\cdot 6$ is the least common multiple of $6$ and $20$.
The order of $10$ is $2$, and indeed $10$ is self-inverse.

\sep

\Def{7.15} For a finite group $G$, $\card{G}$ is called the \emph{order} of 
$G$.

\sep

\Lem{7.6} In a finite group $G$, every element has a finite order.

\Proof Since $G$ is finite, we must have $a^{r}=a^{s}=b$ for some $r$ and $s$
with $r<s$ (and some $b$). Then
\[
a^{s-r}=a^s\cdot a^{-r}=a^s\cdot (a^{r})^{-1} = b \cdot b^{-1} = e.
\]

\sep

\Lem{*} If $G$ is a group and $a\in G$ has finite order, then for any $m\in \Z$
we have
\[
a^{m}=a^{R_{\ord(a)}(m)}.
\]

\sep

\Def{7.6} The smallest subgroup of a group $G$ containing the element $a\in G$
is called the \emph{group generated by} $a$, denoted $\alg{a}$, is defined as
\[
\alg{a}:=\{a^n|n\in\Z\}.
\]

\Com For finite groups we have
\[
\alg{a}:=\{e,a,a^2,\ldots,a^{\ord(a)-1}\}.
\]

\sep

\Def{7.17} A group $G=\alg{g}$ generated by an element $g\in G$ is called
\emph{cyclic} and $g$ is called a \emph{generator} of $G$.

\Com Being cyclic is a special property of a group. Not all groups are
cyclic! A cyclic group can have many generators. In particular, if $g$ is a
generator, then so is $g^{-1}$.

\Ex The group $\alg{\Z_n;\oplus}$ is cyclic for every $n$, where $1$ is a
generator. The generators of $\alg{\Z_n,\oplus}$ are all $g\in\Z_n$ for which
$\gcd(g,n)=1$.

\Ex The additive group of the integers, $\alg{\Z;+,-,0}$, is an infinite cyclic
group generated by 1 and also by $(-1)$. These are the only generators. (Note
that negative powers are allowed according to the definition of a group
generated by an element - this allows the reach all the numbers in $\Z$).

\sep

\Thm{7.7} A cyclic group of order $n$ is isomorphic to $\alg{\Z_n;\oplus}$ (and
hence abelian).

\Com In fact, we use $\alg{\Z_n;\oplus}$ as our standard notation of a cyclic
group of order $n$.

\Proof Let $G=\alg{g}$ be a cyclic group of order $n$ (with neutral element
$e$). The bijection
\[
\Z_n\to G\colon i\mapsto g^{i}
\]
is a group isomorphism, since
\[
i\oplus j \mapsto g^{i+j} = g^{i}*g^{j}, \quad
\ominus i \mapsto g^{-i}, \quad
0\mapsto e.
\]

\Com The two groups are $\alg{\Z_n;\oplus;\ominus;0}$ and $\alg{G;*,^{-1},e}$.

\Com It is easy to see that $g^{i}$ is a generator iff $\gcd(i,n)=1$. (Since
$g^{i}\mapsto i \in \Z$, and $i$ is a generator iff $\gcd(i,n)=1$).

\sep

\todo{Integrate the comments about Diffie-Hellman into the according section.}

\sep

The following theorem is one of the fundamental results in group theory, (sated
without proof):

\Thm{7.8} (Lagrange.) Let $G$ be a finite group and let $H$ be a subgroup of
$G$. Then the order of $H$ divides the order of $G$, i.e., $\card{H}$ divides
$\card{G}$.

The following corollaries are direct applications of Lagrange's theorem.

\Cor{7.9} For a finite group $G$ the order of every elements divides the group
order, i.e., $\ord(a)$ divides $\card{G}$ for every $a\in G$.

\Proof $\alg{a}$ is a subgroup of $G$ of order $\ord(a)$, which according to
Theorem 7.8 must divide $\card{G}$.

\Cor{7.10} Let $G$ be a finite group. Then $a^{\card{G}}=e$ for every $a\in G$.

\Proof According to Corollary 7.9 have $\card{G}=k\cdot \ord(a)$ for some $k$.
Hence
\[
a^{\card{G}} = a^{k\cdot \ord(a)} = (a^{\ord(a)})^{k} = e^{k} = e.
\]

\Cor{7.11} Every group of prime order is cyclic, and in such a group every
element, except the neutral element is a generator.

\Proof Let $\card{G}=p$ with $p$ prime. For any $a$, the order of the subgroup
$\alg{a}$ divides $p$. Thus either $\ord(a)=1$ ord $\ord(a)=$. In the first
case, $a=e$ and in the latter case $G=\alg{a}$.

\Com Groups of prime order play a very important role in cryptography.

\sep

\Def{7.18} $\Z^*_m := \{a \in \Z_m | \gcd(a,m) = 1\}$.

\Com $\alg{\Z_m;\odot,^{-1},1}$ must not necessarily be a group, for example in
$\Z_{12}$ 8 has no inverse. Therefore we defined the subset $\Z_{12}^*$. Such
that every element  $\Z_{12}^*$ has an inverse (in contrast to $\Z_{12}$).

\Def{7.19} The \emph{Euler function } $\varphi\colon \Z^+\to\Z^+$ is defined as
the cardinality of $\Z^*_m$:
\[
\varphi(m)=\card{\Z^*_m}.
\]

\Ex $\Z_{18}=\{1,5,7,11,13,17\}.$ Hence $\varphi(18)=6$.

\Lem{7.12} If $m=\prod_{i=1}^{r}p_i^{e_i}$, then
\[
\varphi(m)=\prod_{i=1}^{r}(p_i-1)p_i^{e_i-1}.
\]
Alternatively, $\varphi(m)$ could be defined as
\[
\varphi(m)=m\cdot \prod_{\substack{p|m\\p
\text{ prime}}}\left(1-\frac{1}{p}\right).
\]


\todo{Understand the proof}

\Thm{7.13} $\alg{\Z^*_m;\odot,^{-1},1}$ is a group.

Now we obtain the following simple but powerful corollary to Theorem 7.8:

\Cor{7.14} (Fermat, Euler). For all $m\geq 2$ and all $a$ with $\gcd(a,m)=1$,
\[
a^{\varphi(m)}\equiv_m 1.
\]
In particular, for every prime $p$ and every $a$ not divisible by $p$,
\[
a^{p-1}\equiv_p 1.
\]

\Thm{7.15} The group $\Z^*_m$ is cyclic if and only if $m=2$, $m=4$, $m=p^{e}$,
or $m=2p^e$, where $p$ is an odd prime and $e\geq 1$.

\sep

\todo{Understand pages 119 - 123 including proof and theorems. Finish copying.}

\subsection*{7.4 Rings and Fields}

\todo{don't forget this theorem, it's not on the word document}

\Thm $\Z_p$ is a field if and only if $p$ is prime.

\Proof This follows from our earlier analysis of $\Z^*_p$, namely that
$\Z_p-\{0\}$ is a multiplicative group if and only if $p$ is prime.

\sep

\subsection{Polynomials over Rings and Fields}

Polynomials over a field $F$ are of special interest since they have properties
in common with the integers, $\Z$.

\sep

\Def[Polynomial] $a(x)$ over a ring $R$ or field $F$ in the indeterminate
$x$ is a formal expression of the form
\[
a(x) = a_d x^d + a_{d-1}x^{d-1} + \cdots + a_1x + a_0 = \sum_{i=0}^{d} a_i x^i
\]
for some non-negative integer $d$. The \emph{degree} $\deg(a(x))$ of $a(x)$ is
the greatest $i$ for which $a_i\neq 0$. The special polynomial $0$ (i.e., all
the $a_i$ are 0) is defined to have degree ``minus infinity''. Let $R[x]$ denote
the set of polynomials (in $x$) over $R$.

\sep

\Lem For two polynomials $p(x)$ and $q(x)$ over a ring $R$ the product of two
polynomials is at most the sum of the degrees:
\[
\deg(p(x)\cdot q(x))\leq\deg(p(x)) + \deg(q(x)),
\]
and the equality holds if $R$ is an integral domain
\[
\deg(p(x)\cdot q(x))=\deg(p(x)) + \deg(q(x)),
\]
And in every case, the degree of the sum is:
\[
\deg(p(x)+q(x)) \leq \max\{\deg(p(x)),\deg(q(x))\}.
\]

\sep

\Thm The polynomials over a ring $R$, denoted $R[x]$, are again a ring with
respect to polynomial addition and multiplication.

\sep

\Lem If $D$ is an integral domain, then
  so is $D[x]$. The units of $D[x]$ are the constant polynomials that are units
  of $D$: $D[x]^* = D^*$.

\sep

For $a,b\in \Z$,
\[
b|a \Longleftrightarrow -b|a,
\]
The analogy for polynomials is as follows:
\[
b(x)|a(x) \Longleftrightarrow \forall v\in F, v\neq 0 : v \cdot b(x)|a(x) 
\]
because
\[
a(x)=b(x)\cdot c(x) \Longrightarrow a(x)=v b(x) \cdot \left(v^{-1} c(x)\right).
\]
Among the polynomials $vb(x)$ (for $v\in F$) there is a distinguished one,
namely that with leading coefficient $1$. This is similar to $b$ and $-b$ being
associated in $\Z$ (see section 7.5.3) and the positive one being distinguished.

\sep

\Def[Monic Polynomial] A polynomial $a(x)\in F[x]$ is called \emph{monic} if the
leading coefficient is 1.

\sep

\Def[Irreducible Polynomial] A polynomial $a(x)\in F[x]$ with degree at least
$1$ is called \emph{irreducible} if it is divisible only by constant polynomials
and by constant multiples of $a(x)$.

\Com The notion of irreducibility in $F[x]$	corresponds to the notion of
primality in $\Z$.

\sep

\Lem A polynomial of degree $d$ may be irreducible or reducible. It can be
checked by testing all irreducible polynomials of degree $\leq d/2$ as possible
divisors (but it may also be irreducible).

Actually, it suffices to test only the monic polynomials because one could
always multiply a divisor by a constant. This irreducibility test is very
similar to the primality test which checks all divisors up to the square root of
the number to be tested.

\sep

Not only the concepts of divisors and divison with remainders carres over from
$\Z$ to $F[x]$, also the concept of a greatest common divisor can be carreid
over:

\Def{7.30} For polynomials $a(x)$ and $b(x)$ in $F[x]$ (not both $0$), a
polynomial $d(x)$ is called \emph{a greatest commen divisor} of $a(x)$ and
$b(x)$ if $d(x)|a(x)$ and $d(x)|b(x)$ and if every common divisor of $a(x)$ and
$b(x)$ divides $d(x)$.

Moreover the monic polynomial $g(x)$ of largest degree such that $g(x)|a(x)$ and
$g(x)|b(x)$ is called \emph{the} greatest common divisor of $a(x)$ and $b(x)$,
denoted $\gcd(a(x),b(x))$.

\todo{Algorithm example}

\sep

\subsubsection*{7.5.2 The Division Property in $F[x]$}

Let $F$ be a field. The ring $F[x]$ has strong similarities with the integers
$\Z$. Both these integral domains have the special property that one can divide
one element $a$ by another element $b\neq 0$, resulting in a quiotient $q$ and a
remainder $r$ which are unique when $r$ is required to be ``smaller'' than the
divisor. In case of the integers, the ``soze'' of $b\in \Z$ is given by the
absolute value $\abs{b}$, and the size of a polynomial $b(x)\in F[x]$ can be
defined as its degree $\deg(b(x))$.

\sep

\Thm{7.26} Let $F$ be a field. FOr any $a(x)$ and $b(x)\neq 0$ in $F[x]$ there
exist unique $q(x)$ (the quotient) and $r(x)$ (the remainder) such that
\[
a(x)=b(x)\cdot q(x) + r(x) \quad \text{and} \quad \deg(r(x)) < \deg(b(x))
\]

\sep

\todo{optional section? - have a look.}

\sep

\subsubsection*{7.5.4 Polynomials as Functions}

For a ring $R$, a polynomial $a(x)\in R[x]$ can be interpreted as a function
$R\to R$ by defining \emph{evaluation} of $a(x)$ at $\alpha\in R$ in the usual
manner. This defines a function $R \to R \colon \alpha \mapsto a(\alpha)$.

\sep

\Def{7.34} Let $a(x)\in R[x]$. An element $\alpha \in R$ for which $a(\alpha)=0$
is called a \emph{root} of $a(x)$.

\Lem{7.29} For a field $F$, $\alpha \in F$ is a root of $a(x)$ if and only if
$(x-\alpha)$ divides $a(x)$.

\Lem{*} Lemma 2.29 implies that an irreducible polynomial of degree $\geq 2	$
has no roots.

\Cor{7.30} A polynomial $a(x)$ of degree $2$ or $3$ over a field $F$ is
irreducible if and only if it has no root.

\todo{check proof}

\Def{7.35} If $\alpha$ is a root of $a(x)$, then its \emph{multiplicity} is the
highest power of $(x-\alpha)$ dividing $a(x)$.

\Thm{7.31} For an integral domain (and hence also a field) $D$, a nonzero
polynomial $a(x)\in D[x]$ of degree $d$ has at most $d$ roots, counting
multiplicities.

\todo{check proof}

\subsubsection*{7.5.5 Polynomial Interpolation}

\sep

\Lem[Interpolation Property] A polynomial $a(x)\in F[x]$ of degree at most $d$
is uniquely determined by any $d+1$ values of $a(x)$, i.e., by
$a(\alpha_1),\ldots,a(\alpha_{d+1})$ for any distinct $a_1,\ldots,a_{d+1}\in F$.

\textbf{Lagrange's Interpolation Formula} Let 
\begin{align*}
a(\alpha_1) &= \beta_1\\
a(\alpha_2) &= \beta_2\\
&\vdots\\
a(\alpha_{d+1}) &= \beta_{d+1}
\end{align*}
Then $a(x)$ is given by Lagrange's interpolation formula:
\[
a(x) = \sum_{i=1}^{d+1} \beta_i u_i(x)
\]
where $u_i(x)$ is given by:
\[
u_i(x) = \frac{(x-\alpha_1)\cdots (x-\alpha_{i-1})(x-\alpha_{i+1})\cdots
(x-\alpha_{d+1})}{(\alpha_i -
\alpha_1)\cdots(\alpha_i-\alpha_{i-1})(\alpha_i-\alpha_{i+1})(\alpha_i-\alpha_{d+1})}.
\]

% TODO: useful comment
\begin{comment}
Hence, in $u_i(x)$, the factors, $(x-\alpha_i)$ in the
numerator, and the factor $(\alpha_i-\alpha_i)$ (which would be 0) in the
nominator, are left out.

Note that for $u_i(x)$ to be well-defined, all constant terms $\alpha_i -
\alpha_j$ in the denominator must be invertible. This is guaranteed if $F$ is a
field since $\alpha_i - \alpha_j\neq 0$ for $i\neq j$. Note also that the
denominator is simply a constant and hence $u_i(x)$ is indeed a polynomial of
degree $d$. It eas easy to verify that $u_i(\alpha_i)=1$ and $u_i(\alpha_j)=0$
for $j\neq i$. Thus the polynomials $a(x)$ and $\sum_{i=1}^{d+1}\beta_i u_i(x)$
agree when evaluated at any $\alpha_i$. Note that $a(x)$ has degree at most $d$.

It remains to prove the uniqueness. Suppose there is another polynomial $a'(x)$
of degree at most $d$ such that $\beta_i=a'(\alpha_i)$ for $i=1,\ldots,d+1$.
Then $a(x)-a'(x)$ is also a polynomial of degree at most $d$, which (according
to Theorem 7.31) can have at most $d$ roots, unless it is 0. But $a(x)-a'(x)$
has indeed the $d+1$ roots $\alpha_1,\ldots,\alpha_{d+1}$. Thus it must be 0,
which implies $\alpha(x) = \alpha'(x)$.
\end{comment}

% TODO: useful example
\begin{comment}
\Ex Let $a(x)$ be a polynomial of degree $4$ over $GF(7)[x]$, of which we know
that $a(x)$ has a double root at $x=2$. Further, we know that $a(3)=2$, $a(4)=3$
and $a(6)=5$. Now we want to determine $a(0)$.

First we need to determine $a(x)$. Since $2$ is a double root, $a(x)=(x-2)^2
b(x)$, where $b(x)$ is a polynomial of degree 2.
We know that
\begin{align*}
a(3)=(3-2)^2\cdot b(3)=2 & \Longleftrightarrow b(3)=2 \\
a(4)=(4-2)^2\cdot b(4)=3 & \Longleftrightarrow b(4)= 4^{-1}\cdot 3 = 2 \cdot 3 =6 \\ 
a(6)=(6-2)^2\cdot b(6)=5 & \Longleftrightarrow b(6) = 2^{-1}\cdot 5 = 4
\cdot 5 = 6
\end{align*}
Now we use the Lagrange interpolation to determine $b(x)$:
\begin{align*}
b(x) &= 2 \tfrac{(x-4)(x-6)}{(3-4)(3-6)} + 6 \tfrac{(x-3)(x-6)}{(4-3)(4-6)} +
6\tfrac{(x-3)(x-4)}{(6-3)(6-4)}\\
&= 2 \tfrac{(x-4)(x-6)}{(-1)(-3)} + 6 \tfrac{(x-3)(x-6)}{(-2)} +
6\tfrac{(x-3)(x-4)}{3\cdot 2}\\
&= 2 \tfrac{(x-4)(x-6)}{3} + 6 \tfrac{(x-3)(x-6)}{5} +
6\tfrac{(x-3)(x-4)}{6}\\
&= 2\cdot 3^{-1}(x-4)(x-6)+ 6 \cdot 5^{-1}(x-3)(x-6) +
6 \cdot 6^{-1}(x-3)(x-4)\\
&= 2\cdot 5(x-4)(x-6)+ 6 \cdot 3(x-3)(x-6) + (x-3)(x-4)\\
&= 3(x-4)(x-6)+ 4(x-3)(x-6) + (x-3)(x-4)\\
&= 3(x+3)(x+1)+ 4(x+4)(x+1) + (x+4)(x+3)\\
&= 3(x^2+4x+3)+ 4(x^2+5x+4) + x^2+5\\
&= x^2+4x+2\\
\end{align*}
This gives us:
\begin{align*}
a(x)
&=(x-2)^2 b(x)\\
&=(x+5)^2 (x^2+4x+2)\\
&=(x^2+10x+25) (x^2+4x+2)\\
&=(x^2+3x+4) (x^2+4x+2)\\
&=x^4+7x^3+18x^2+22x+8\\
&=x^4+4x^2+x+1\\
\end{align*}
Therefore $a(0)=1$.
\end{comment}



\subsection{Finite Fields}

So far we have seen the finite field $GF(p)$, where $p$ is prime. In this
section we discuss all remaining finite fields.

\subsubsection*{7.6.1 The Ring $F[x]_{m(x)}$}

We continue to explore the analogies between the rings $\Z$ and $F[x]$. In the
same way as we can compute in the integers $\Z$ modulo an integer $m$, yielding
the ring $\alg{\Z_m;\oplus,\ominus,0,\odot, 1}$, we can also compute in $F[x]$
modulo a polynomial $m(x)$. Let $R_{m(x)}(a(x))$ denote the (unique) remainder
when $a(x)$ is divided by $m(x)$. The concept of congruence modulo $m(x)$ is
defined like congruence modulo $m$. For $a(x),b(x)\in F[x]$,
\[
a(x)\equiv_{m(x)} b(x) :\Longleftrightarrow m(x) | \left(a(x) - b(x)\right).
\]

\sep

\Lem{7.33} Congruence modulo $m(x)$ is an equivalence relation on $F[x]$, and
each equivalence class has a unique representative of degree less than
$\deg(m(x))$.

\Proof Analogous to proof that congruence modulo $m$ is an equivalence relation
on $\Z$.

\sep

\Def{7.36} Let $m(x)$ be a polynomial of degree $d$ over $F$. Then 
\[
F[x]_{m(x)}:=\{a(x)\in F[x] | \deg(a(x)) < d\}.
\]

\sep

\Lem{7.34} Let $F$ be a finite field with $q$ elements and let $m(x)$ be a
polynomial of degree $d$ over $F$. Then $\card{F[x]_{m(x)}}=q^d$.

\sep

\Lem{7.35} $F[x]_{m(x)}$ is a ring with respect to addition and multiplication
modulo $m(x)$.

\Com Note that addition (but not multiplication) in $F[x]$ and $F[x]_{m(x)}$ are
identical, since the sum of two polynomials is never reduced modulo $m(x)$
because the degree of the sum is at most the maximum of the two degrees.

\sep

\Lem{7.36} The congruence equation
\[
a(x)b(x)\equiv_{m(x)}1
\]
(for a given $a(x)$) has a solution $b(x)\in F[x]_{m(x)}$ if and only if
$\gcd(a(x),m(x))=1$. The solution is uniuque - if it exists it is called the
inverse of $a(x)$ modulo $m(x)$. Therefore, we can also define the set of units:
\[
F[x]^*_{m(x)} = \{a(x)\in F[x]_{m(x)} | \gcd(a(x),m(x))=1	\}.
\]

\Com Inverses inf $F[x]^*_{m(x)}$ can be computed efficiently, but we do not
discuss an algorithm here.

\sep

\subsubsection*{7.6.2 Constructing Extension Fields}

The following theorem is analogous to Theorem 7.23 stating that $\Z_m$ is a
field if and only if $m$ is prime.

\Thm{7.37} The ring $F[x]_{m(x)}$ is a field if and only if $m(x)$ is
irreducible.

\Com $F[x]_{m(x)}$ is called an \emph{extension field} of $F$.

\todo{Understand proof!}

\subsection*{7.7 Application: Error-Correcting Codes}









\sep

\textbf{Polynomial Division in a Field}

Here we show how to divide $3x^5 + 9x^2+4x+7$ by $2x^2+x+5$ in $GF(13)$.

First we write the multiplication table with the \emph{coefficients of the
divisor as column titles} and the \emph{elements of $GF(13)$ as rows}: (Don't
forget to think about the spaces if the divisor does not contain all descending powers of $x$.)

\begin{tabular}{c|c|c|c}
   & $x^2$ & $x$ & $c$\\
GF(13)   & 2  &  1 &  5 \\
\hline
 0 &  0 &  0 &  0 \\
 1 &  2 &  1 &  5 \\
 2 &  4 &  2 & 10 \\
 3 &  6 &  3 &  2 \\
 4 &  8 &  4 &  7 \\
 5 & 10 &  5 & 12 \\
 6 & 12 &  6 &  4 \\
 7 &  1 &  7 &  9 \\
 
 \textcolor{green}{8} &  \colorbox{yellow}{\textcolor{red}{3}} & 
 \textcolor{red}{8} &
 \textcolor{red}{1}
 \\
 9 &  5 &  9 &  6 \\
10 &  7 & 10 & 11 \\
11 &  9 & 11 &  3 \\
12 & 11 & 12 &  8 \\
\end{tabular}

Then we write the polynomials divisor and dividend next to each other. It's
important to write the dividend with 0 coefficients for the powers that are not
existing, such that we'll have space to write our steps below.

Now we're ready to execute the polynomial division. We'll always ask us
\emph{what times the leading coefficient of the divisor is equal to the leading
coefficient of the dividend (or the remaining dividend)}. We can find out this
easily as follows, e.g.: 2 times what gives 3? - In the column of the leading
coefficient (in this case 2), wee look where 3 occurs, then we get the
corresponding number at the left of the row: in this case 8 and write it to the
top. Furthermore, copy the contents of the rest of the row for the
subtraction. Then we just do our subtraction in $GF(13)$ until we get the
quotient and the remainder:

\[
\begin{array}{lllllll}
         & \ \ \ \ 8x^3 &+9x^2 & \textcolor{green}{+8}x &+4\\
\cline{2-7}
2x^2+x+5 & | \ \ 3x^5 &+0x^4 &+0x^3 &+9x^2 & +4x & +7\\
         & -3x^5 & -8x^4 & -1x^3\\
\cline{2-4}
         &       & +5x^4 & +12x^3 & +9x^2\\
         &       & -5x^4 & -9x^3  & -6x^2\\
\cline{3-5}
         &       &       & \colorbox{yellow}{+3}x^3  & +3x^2 & +4x\\
         &       &       & \textcolor{red}{-3}x^3  & \textcolor{red}{-8}x^2
         & \textcolor{red}{-1}x\\
\cline{4-6}
         &       &       &        & +8x^2 & +3x & +7\\
         &       &       &        & -8x^2 & -4x & -7\\
\cline{5-7}
         &       &       &        &       & +12x\\
                         
\end{array}
\]

We'll stop with the division as soon as the degree of the remainder is less than
the degree of the divisor.

Finally, the division yields:
\[
q(x) = 8x^3+9x^2+8x+4 \quad \text{(quotient)}
\]
\[
r(x) = 12x \quad \text{(remainder)}
\]

\sep

\textbf{Factoring a Polynomial in a Field}

Lets assume we want to factor the polynomial $x^3+4x+4$ over $GF(7)$. Then we do
the following: Since the degree of the polynomial is $\in\{2,3\}$ we know that
is reducible iff it has a root. Thus we try all numbers in $GF(7)$, e.g.
$\{0,\ldots,6\}$ and take the first matching candidate.

For 4: $4^3+4\cdot 4 + 4 = 2^6 +16 + 4 = 64 + 16 + 4 = 84 \equiv_7 0$.

As we can see $4$ is a root, thus we deflate this root from the polynomial by
dividing with $(x-4)$. The division becomes even easier if we divide by
$(x+3)=(x-4)$, since we're in $GF(7)$.

We'll just divide as described and get.
\[
(x^3+4x+4) : (x+3) = (x^2+4x+6)
\]
Then we get a polynomial of degree 2 : $(x^2+4x+6)$. Trying all the numbers from
0 to 6 shows us that this polynomial of degree 2 is not further reducible. This
gives us the final factorisation:
\[
(x^3+4x+4) = (x+3)(x^2+4x+6).
\]

\sep

\textbf{Irreducible Polynomials over $GF(2)$ of degree $n$:}

\begin{tabular}{lp{.4\textwidth}}
n	& irreducible polynomials\\
1	& $x+1, x$\\
2	& $x^2+x+1$\\
3	& $x^3+x+1$, $x^3+x^2+1$\\
4	& $x^4+x+1$, $x^4+x^3+x^2+x+1$,  $x^4+x^3+1$\\
5   & $x^5+x^2+1$, $x^5+x^3+x^2+x+1$, $x^5+x^3+1$, $x^5+x^4+x^3+x+1$,
$x^5+x^4+x^3+x^2+1$, $x^5+x^4+x^2+x+1$
\end{tabular}

\sep

\textbf{Irreducible Polynomials over $GF(3)$ of degree $n$}:
\begin{tabular}{lp{.4\textwidth}}
n	& irreducible polynomials\\
2	& $x^2 + 2x + 2$, $x^2 + 1$, $x^2 + x + 2$\\
3	& $x^3 + 2x + 1$,
$x^3 + x^2 + x + 2$, 
$x^3 + x^2 + 2$, 
$x^3 + 2x^2 + x + 1$, 
$x^3 + x^2 + 2x + 1$, 
$x^3 + 2x^2 + 2x + 2$, 
$x^3 + 2x + 2$, 
$x^3 + 2x^2 + 1$, 
\end{tabular}

\sep

\textbf{Multiplicative Inverses over Finite Fields}

\begin{tabular}{|c|c|c|c|c|c|c|}
\toprule
& \multicolumn{6}{c|}{$GF(n)$}
\\
          & 2 & 3 & 5 & 7 & 11 &  13\\
\midrule
$1^{-1}$  & 1 & 1 & 1 & 1 &  1 &   1\\
\hline
$2^{-1}$  &   & 2 & 3 & 4 &  6 &   7\\
\hline
$3^{-1}$  &   &   & 2 & 5 &  4 &   9\\
\hline
$4^{-1}$  &   &   & 4 & 2 &  3 &  10\\
\hline
$5^{-1}$  &   &   &   & 3 &  9 &   8\\
\hline
$6^{-1}$  &   &   &   & 6 &  2 &  11\\
\hline
$7^{-1}$  &   &   &   &   &  8 &   2\\
\hline
$8^{-1}$  &   &   &   &   &  7 &   5\\
\hline 
$9^{-1}$  &   &   &   &   &  5 &   3\\
\hline
$10^{-1}$  &   &   &   &   & 10 &  4\\
\hline
$11^{-1}$  &   &   &   &   &    &  6\\
\hline
$12^{-1}$  &   &   &   &   &    & 12\\
\bottomrule
\end{tabular}

\Com $GF(n)$ is isomporph to $\Z_n$. But for the multiplicative group we just
use $\Z_n-\{0\}$ (without $0$, since it has no inverse). Note that in most of
the cases $\Z_n-\{0\}=\Z^*_n$. Further note that there is no field of size
1.

\todo{Note to say what you mean by in most of the cases, if $n$ is prime its
clear. What happens if $n$ is a power of an odd prime? }

\sep

\textbf{Multiplication in $GF(2)[x]_{x^2+x+1}$}

\begin{tabular}{c|cccc}
$\cdot$ & $0$ & $1$ & $x$ & $x+1$\\
\hline
0 & 0 & 0 & 0 & 0\\
1 & 0 & 1 & $x$ & $x+1$\\
$x$ & 0 & $x$ & $x+1$ & 1\\
$x+1$ & 0 & $x+1$ & 1 & $x$
\end{tabular}

\sep

\textbf{Multiplication in $GF(2)[x]_{x^3+x+1}$}

\todo{Table needed?}

\sep

\textbf{Multiplication in $GF(3)[x]_{x^2+1}$}

\todo{Table needed?}
